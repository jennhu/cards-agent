\documentclass[11pt]{article}

% BASIC PACKAGES
\usepackage{latexsym}
\usepackage{amssymb,amsmath,pgf,graphicx}
\usepackage{amsthm}
\usepackage{thmtools}

% FONTS AND SPACING
\usepackage{amsfonts}
\usepackage{tgpagella} % font
\usepackage[margin=0.8in]{geometry}
\usepackage{setspace}
%\onehalfspacing % or
\doublespacing

% SPECIAL FORMATTING
\usepackage{fancyhdr}
\usepackage[T1]{tipa} % for ipa
\usepackage{color}
\usepackage{tabulary}
\usepackage{mdframed}
\usepackage{algpseudocode}
\usepackage{tikz}
\usepackage{tkz-graph} % for graphs

% MATHEMATICAL COMMANDS
\DeclareMathOperator*{\argmax}{arg\,max}
\newcommand{\bvec}[1]{\vec{\mathbf{#1}}}
\newcommand{\R}{\mathbb{R}}
\newcommand{\dist}{\textsc{Distance}}
\newcommand{\overlap}{\textsc{Overlap}}
\newcommand{\lkhd}{\textsc{Likelihood}}

% DEFINITIONS AND THEOREMS
\newtheorem{theorem}{Theorem}
\declaretheorem[style=definition,qed=$\blacksquare$]{definition}

\pagestyle{fancy}
\fancyhf{}
\rhead{\today}
\lhead{Jennifer Hu}
\cfoot{\thepage}

\begin{document}

\begin{center}
\Large{Finding optimal goals in card games with uncertainty}
\end{center}

\section{Introduction}

Our central question is the following: given the history of the game, what is the optimal goal to pursue?

There are 52 possible goals (i.e. straight flushes of 6 cards). We give them each different weights in the weight vector $\bvec{w} \in \R^{52}$, which is updated every time we learn new information about the game. At the beginning before any cards are dealt, every goal is equally (un)optimal, and $\bvec{w} := \bvec{0}$.

\section{The Metrics}

I propose to evaluate goals with three metrics: $\overlap$, $\dist$, and $\lkhd$.

\subsection{Overlap}

$\overlap$ measures how many cards in common a certain goal $G$ has with a set of cards $C$, typically the ones visible (in hands and on the table). Formally, $\overlap$ is defined as the following:
\begin{equation}
\overlap(C,G) = |C \cap G|
\label{eq:overlap} \end{equation}
$\overlap$ takes on values in the interval $[0,6]$. It is independent of game history.

\subsection{Distance}

$\dist$ measures how close a certain goal $G$ is from a set of cards $C$. It is independent of game history.
% TODO

IS DISTANCE JUST 6 - OVERLAP?!?!?!?!

\subsection{Likelihood}

$\lkhd$ measures how likely a certain goal $G$ can be obtained given all previous history $H$. $\lkhd$ is defined as the following:
\begin{align}
  \mathcal{L}(G,H) &= \sum_{g \in G} (1 - P(g \text{ has been discarded}|H))
\end{align}

Given a card $g$ and history $H$, we can find $P(g \text{ has been discarded}|H)$ in the following way. First, we assume that the history $H$ contains information about $r_i$, the number of cards reshuffled at round $i$ for all $i$, as well as which cards we have seen and not seen. Let $s$ be the index of the round that $g$ was last seen, let $n$ be the number of rounds that have occurred between $s$ and now, and let $D_i$ be the size of the deck at round $i$. Then we have:
\begin{equation}
P(g \text{ has been discarded}|H) = \begin{cases}
  0 & g \in \text{ hands or table, or unseen} \\
  \frac{4-r_s}{4-r_s+r_s\prod_{j=1}^n(1-4/{D_{s+j}})} & \text{o.w.}
\end{cases}
\label{eq:p-discarded} \end{equation}

\begin{proof}
  It is easy to see that a card $g$ cannot have been discarded if it is currently in the hands, on the table, or unseen (in the deck). Now, consider the case where $g$ has not been seen for $n$ rounds (since round $s$). Let $F$ be the event that $g$ was discarded at around $s$, and let $U$ be the event that we haven't seen $g$ for $n$ rounds. By Bayes' Rule, we have:
\begin{align}
  P(F|U) &= \frac{P(U|F)P(F)}{P(U)} \\
  &= \frac{(1)\left(\frac{4-r_s}{4}\right)}{P(U|F)P(F) + P(U|F^c)P(F^c)} \\
  &= \frac{(4-r_s)/4}{(4-r_s)/4 + (r_s/4)P(U|F^c)}
\end{align}

Let's look at $P(U|F^c)$. Since we are conditioning on $F^c$, we know that $g$ was reshuffled in the previous round, and the size of the deck has been updated to $D$. The probability of not drawing $g$ in this round is
\begin{equation}
  \frac{{D-1 \choose 4}}{{D \choose 4}} = \frac{D-4}{D},
\end{equation} and this pattern continues for every subsequent round. This gives us
\begin{equation}
  P(U|F^c) = \prod_{j=1}^n \frac{D_{s+j} - 4}{D_{s+j}}.
\end{equation}

Using this result and simplifying, we finally obtain
\begin{equation}
  P(F|U) = \frac{4-r_s}{4-r_s+r_s\prod_{j=1}^n(1-\frac{4}{{D_{s+j}}})}.
\end{equation}
\end{proof}

Note that we can also write the size of the deck at round $k$ as
\begin{equation}
  D_k = 46 + \sum_{i=1}^{k-1} r_i - 4k.
\label{eq:decksize} \end{equation}

\begin{proof}
  In the base case, $k=1$ yields $46-4(1) = 42$, which is correct since we deal 10 cards from the original 52 in the first round. Suppose now that (\ref{eq:decksize}) is true for $k=n$. Then we have:
    \begin{align}
      D_{n+1} &= D_n + r_n - 4 \\
      &= (46 + \sum_{i=1}^{n-1} r_i - 4n) + r_n - 4 \\
      &= 46 + \sum_{i=1}^n r_i - 4(n+1)
    \end{align}
  By induction, (\ref{eq:decksize}) holds for all values of $k$.
\end{proof}

\section{Evaluating Goals}

Every time the history $H$ is updated and we observe a new set of cards $C$, the weights $w_i$ for each goal $G_i$ are updated with a linear combination of the three metrics:
\begin{equation}
  w_i := \alpha_0 \cdot \overlap(G_i,C) + \alpha_1 \cdot \dist(G_i,C) + \alpha_2 \cdot \lkhd(G_i,H)
\label{eq:w-update} \end{equation}

The values of $\{\alpha\}$ depend on how we want to weight the different metrics. A less rational player might weight $\overlap$ more than $\lkhd$, for example.

\end{document}
